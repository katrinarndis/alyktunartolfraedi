% Options for packages loaded elsewhere
\PassOptionsToPackage{unicode}{hyperref}
\PassOptionsToPackage{hyphens}{url}
%
\documentclass[
]{book}
\usepackage{amsmath,amssymb}
\usepackage{lmodern}
\usepackage{iftex}
\ifPDFTeX
  \usepackage[T1]{fontenc}
  \usepackage[utf8]{inputenc}
  \usepackage{textcomp} % provide euro and other symbols
\else % if luatex or xetex
  \usepackage{unicode-math}
  \defaultfontfeatures{Scale=MatchLowercase}
  \defaultfontfeatures[\rmfamily]{Ligatures=TeX,Scale=1}
\fi
% Use upquote if available, for straight quotes in verbatim environments
\IfFileExists{upquote.sty}{\usepackage{upquote}}{}
\IfFileExists{microtype.sty}{% use microtype if available
  \usepackage[]{microtype}
  \UseMicrotypeSet[protrusion]{basicmath} % disable protrusion for tt fonts
}{}
\makeatletter
\@ifundefined{KOMAClassName}{% if non-KOMA class
  \IfFileExists{parskip.sty}{%
    \usepackage{parskip}
  }{% else
    \setlength{\parindent}{0pt}
    \setlength{\parskip}{6pt plus 2pt minus 1pt}}
}{% if KOMA class
  \KOMAoptions{parskip=half}}
\makeatother
\usepackage{xcolor}
\usepackage{longtable,booktabs,array}
\usepackage{calc} % for calculating minipage widths
% Correct order of tables after \paragraph or \subparagraph
\usepackage{etoolbox}
\makeatletter
\patchcmd\longtable{\par}{\if@noskipsec\mbox{}\fi\par}{}{}
\makeatother
% Allow footnotes in longtable head/foot
\IfFileExists{footnotehyper.sty}{\usepackage{footnotehyper}}{\usepackage{footnote}}
\makesavenoteenv{longtable}
\usepackage{graphicx}
\makeatletter
\def\maxwidth{\ifdim\Gin@nat@width>\linewidth\linewidth\else\Gin@nat@width\fi}
\def\maxheight{\ifdim\Gin@nat@height>\textheight\textheight\else\Gin@nat@height\fi}
\makeatother
% Scale images if necessary, so that they will not overflow the page
% margins by default, and it is still possible to overwrite the defaults
% using explicit options in \includegraphics[width, height, ...]{}
\setkeys{Gin}{width=\maxwidth,height=\maxheight,keepaspectratio}
% Set default figure placement to htbp
\makeatletter
\def\fps@figure{htbp}
\makeatother
\setlength{\emergencystretch}{3em} % prevent overfull lines
\providecommand{\tightlist}{%
  \setlength{\itemsep}{0pt}\setlength{\parskip}{0pt}}
\setcounter{secnumdepth}{5}
\usepackage{booktabs}
\ifLuaTeX
  \usepackage{selnolig}  % disable illegal ligatures
\fi
\usepackage[]{natbib}
\bibliographystyle{plainnat}
\IfFileExists{bookmark.sty}{\usepackage{bookmark}}{\usepackage{hyperref}}
\IfFileExists{xurl.sty}{\usepackage{xurl}}{} % add URL line breaks if available
\urlstyle{same} % disable monospaced font for URLs
\hypersetup{
  pdftitle={Inngangur að ályktunartölfræði},
  pdfauthor={Katrín Arndís},
  hidelinks,
  pdfcreator={LaTeX via pandoc}}

\title{Inngangur að ályktunartölfræði}
\author{Katrín Arndís}
\date{2023-01-17}

\begin{document}
\maketitle

{
\setcounter{tocdepth}{1}
\tableofcontents
}
\hypertarget{formuxe1li}{%
\chapter{Formáli}\label{formuxe1li}}

Í þessu skjali fer ég yfir nokkur hugtök og hugmyndir sem er gott að kunna skil á. Athugið að þetta er óyfirfarið efni svo villur gætu leynst í textanum.

\hypertarget{grunn-hugtuxf6k}{%
\section{Grunn hugtök}\label{grunn-hugtuxf6k}}

Eftirfarandi eru hugtök sem þarf að þekkja við lestur þessa skjals:

\begin{itemize}
\tightlist
\item
  \textbf{Þýði} er sá hópur sem við viljum læra eitthvað um; það getur
  endurspeglað stóran hóp einstaklinga (Evrópubúar), smærri hóp
  (Íslendingar) eða heldur lítinn hóp (nemendur HR).

  \begin{itemize}
  \tightlist
  \item
    Þýði þarf að vera skýr, afmarkaður hópur einstaklinga sem eiga
    eitthvað sameiginlegt.\footnote{Hér er talað um fólk, því fólk er yfirleitt viðfangsefni
      rannsókna í sálfræði. Það er þó auðvitað hægt að gera rannsóknir og
      beita ályktunartölfræði á dýr, bíla, sveppi og jafnvel hnetusmjör.
      Hugmyndin stendur þó -- þýði er heildin sem við viljum þekkja eða
      draga ályktanir um -- en er gjarna of stór til að það sé raunsætt að
      mæla hvert einasta stak sem tilheyrir því. (Stak er því gjarna notað
      til að lýsa hverri einingu sem þýði samanstendur af, í þýði
      Íslendinga myndi hver einstaklingur vera eitt stak í þýðinu).}
  \end{itemize}
\item
  \textbf{Úrtak} er sá hópur sem við mælum því við teljum hann endurspegla
  þýðið. Við reynum yfirleitt að draga með tilviljunarkenndum hætti úr þýði.
\item
  \textbf{Þýðistölur} eru þau gildi sem myndu fást ef við hefðum aðgang að
  öllu þýðinu.

  \begin{itemize}
  \tightlist
  \item
    Gefum okkur að við viljum vita hæð Íslendinga: Það er einhvert raunverulegt gildi sem fengist ef \textbf{öll stök þýðis} væru mæld. Það er þó sjaldnast hægt að mæla \emph{alla} sem þýði samanstendur af -- við drögum því úrtak og athugum eiginleika úrtaksins til að komast sem næst raunverulegu þýðisgildi.
  \end{itemize}
\item
  \textbf{Úrtakstölur} eru þau gildi sem fást frá úrtakinu. Í úrtaki Íslendinga fæst einhver meðalhæð. Þessi meðalhæð úrtaksins er dæmi um úrtakstölu -- hún er það næsta sem við komumst því að vita hver hæðin er í \emph{raunveruleikanum (þýðinu)}.
\end{itemize}

\hypertarget{duxe6mi}{%
\subsection*{Dæmi}\label{duxe6mi}}
\addcontentsline{toc}{subsection}{Dæmi}

\emph{Við viljum vita hvort aukinn menntun skili sér í hærri launum á Íslandi.}

Hér er þýðið okkar afmarkað -- við erum að skoða fólk með búsetu á Íslandi. Úr því þýði myndum við draga úrtak, síðan myndum við athuga fjölda ára sem hver einstaklingur úrtaksins hefur verið í námi og hver mánaðarlaun viðkomandi eru.

Spurningin okkar er einnig skýr og við gætum endurorðað hana á eftirfarandi hátt: Eru mánaðarlaun að jafnaði hærri eftir því sem fjöldi ára í námi eykst? Við getum einfaldað spurninguna enn meira og athugað hvort það sé munur á meðallaunum þeirra sem hafa lokið háskólamenntun og þeirra sem hafa ekki lokið neinni háskólamenntun.

\hypertarget{tilguxe1tupruxf3fun}{%
\chapter{Tilgátuprófun}\label{tilguxe1tupruxf3fun}}

Þegar við setjum upp rannsóknir eða söfnum gögnum þá höfum við
einhverjar spurningar í huga sem við viljum leitast svara við.

\textbf{Tilgátuprófun} er það ferli að setja spurninguna okkar fram í formi
tilgáta og nota viðeigandi tölfræðipróf til að fá einfalt já/nei við
spurningunni okkar eftir því hvort niðurstöður reynist marktækar eða
ómarktækar.

Tilgátuprófun svarar spurningunni \textbf{hvort það sé munur} -\/- í kjölfarið
viljum við auðvitað vita \textbf{hve miklu munar} og notum þá spá, þar sem
úrtakstölur eru notaðar til að spá fyrir um þýðisgildin. Augljóst
vandamál er að tölfræðiforrit skilja ekki setningar - við gætum ekki
beinlínis sett gögn inn í úrvinnslu ásamt spurningunni ``eru mánaðarlaun
ólík eftir menntun fólks?''.

Það má því ímynda sér ferlið í grófum dráttum á eftirfarandi hátt: Við
yfirfærum spurninguna okkar yfir á \emph{tölfræðilegt form} sem forritið
skilur og segjum því hvaða tölfræðiaðferð það eigi að nota við að svara
spurningunni okkar. Forritið vinnur úr gögnunum fyrir okkur og spýtir út
úr sér niðurstöðum. Þessar niðurstöður eru þá enn á \emph{tölfræðilegu formi}
og það er okkar verk að skilja hvað niðurstöðurnar segja okkur og geta
gefið túlkun sem setur þær aftur á \emph{mannamál}.

\hypertarget{cross}{%
\section{Tilgátur}\label{cross}}

Við byrjum á því að setja fram \textbf{tilgátur}. Í okkar dæmi viljum við
vita hvort það sé munur á launum einstaklings eftir menntun viðkomandi.

\hypertarget{auxf0altilguxe1ta}{%
\subsection{Aðaltilgáta}\label{auxf0altilguxe1ta}}

\textbf{Aðaltilgáta} \footnote{Einnig kölluð gagntilgáta, rannsóknartilgáta, (e.
  alternative hypothesis)} er sú tilgáta sem tilgreinir
einhvers konar mun. Vandamálið er að hún tilgreinir enga fasta tölu sem
hægt er að prófa og því ógerlegt að prófa hana beint. Í okkar dæmi gæti
sannarlega verið munur á launum eftir menntun, en hve mikinn mun ætti að
prófa? Fyrir hvert ár í námi væri hægt að prófa hvort laun hækki um 1kr,
10kr, 15kr, 100kr, 1.000kr, 35.000kr o.s.frv. Tilgátan ``það er munur''
hefur ótal möguleika en úrvinnslan krefst þess að við tilgreinum fast
gildi sem hægt er að prófa. Við setjum því fram núlltilgátu.

\hypertarget{nuxfalltilguxe1ta}{%
\subsection{Núlltilgáta}\label{nuxfalltilguxe1ta}}

\textbf{Núlltilgáta} tilgreinir tiltekið tölugildi, þar sem \emph{ef} það tiltekna
gildi væri í raun rétt -- þá gæti aðaltilgátan ekki líka verið rétt. Í
okkar dæmi yrði núlltilgátan sú að laun séu þau sömu óháð
menntunarstigi. Það er ekki það sama og að segja að \emph{allir séu með sömu
laun} heldur aðeins að ef þú berð saman meðallaun einstaklinga sem hafa
ólíka menntun þá sé munurinn á meðallaunum þeirra = 0 (engin munur á
meðallaunum þeirra sem hafa verið fá ár í skóla og þeirra sem hafa verið
mörg ár í skóla). Ef niðurstaða er sú að núlltilgátan sé sennilega röng
-- þá hlýtur andstæða hennar að vera rétt. M.ö.o. ef munur á launum
eftir menntun er \textbf{ekki} núll, þá er jú \emph{einhver} munur og aðaltilgátan
hlýtur því að vera rétt.

\begin{center}\rule{0.5\linewidth}{0.5pt}\end{center}

Nú höfum við sett fram tilgátur og næsta skref er að athuga \textbf{hvort}
við höfum rétt fyrir okkur -- til þess framkvæmum við marktektarpróf.

\hypertarget{marktekt}{%
\chapter{Marktekt}\label{marktekt}}

Við gætum skoðað úrtakstölurnar okkar og séð að í úrtakinu séu meðallaun þeirra sem hafa litla menntun ólík þeirra sem hafa mikla menntun. Tilgangur þess að beita ályktunartölfræði er þó ekki til að \emph{spá fyrir} um úrtakið okkar -- úrtakstölurnar eru jú beint fyrir framan okkur, lýsandi tölfræði myndi því duga til að lýsa þeim. Sá munur sem við sjáum í úrtakinu okkar gæti þó verið tilkominn af hreinni tilviljun. Ímyndum okkur tvö dæmi:

\begin{enumerate}
\def\labelenumi{\arabic{enumi}.}
\tightlist
\item
  20 manna úrtak þar sem launamunur reynist 10kr á milli þeirra sem hafa háskólamenntun og þeirra sem hafa ekki háskólamenntun. Í þessu úrtaki er vissulega munur en það er einnig auðvelt að ímynda okkur að m.v. úrtakstölur, þá sé í raun engin munur á launum hópana í þýði. Þ.e. þessi 10kr munur endurspeglar bara þá tilviljun sem má búast við þegar við drögum úrtak.
\item
  2000 manna úrtak þar sem launamunur reynist 100.000kr á milli hópanna. Hér þætti okkur frekar ólíklegt að svo mikill munur fyndist, af tilviljun, í svo stóru úrtaki \textbf{ef það væri í raun engin munur á hópunum í þýði}.
\end{enumerate}

Eftir því sem munurinn er meiri og úrtakið stærra, þeim mun ólíklegra verður að þykja að þessar úrtakstölur séu einskær tilviljun. Þá situr eftir sú spurning -- hversu mikill munur er nógu mikill? Og þá miðað við hversu stórt úrtak? \footnote{Ásamt fleiri spurningum sem marktektarpróf taka til greina við útreikning, þetta er einföldun.}

Marktektarpróf tekur tillit til ýmissa upplýsinga um úrtakið okkar og athugar líkur þess að fá okkar úrtakstölur \textbf{ef þetta úrtak kæmi úr þýði þar sem núlltilgátan er í raun rétt.} \emph{Ef það er í raun \textbf{enginn} munur á launum þeirra sem hafa og hafa ekki háskólamenntun, hverjar eru þá líkurnar á því að fá 100.000kr mun í úrtaki með 2.000 manns?}

Ef líkurnar eru mjög litlar, þá ályktum við að það sé ólíklegt að núlltilgátan sé rétt. Ef við náum að hafna núlltilgátunni, þá tökum við upp aðaltilgátuna og ályktum þ.a.l. að það sé í raun munur á hópunum í þýði. Ef líkurnar eru miklar -- t.d. ef niðurstöður segja okkur að það séu 50\% líkur á að fá þessar úrtakstölur ef við drógum það af tilviljun úr þýði þar sem það er í raun enginn munur -- þá myndum við hika verulega við að henda fram fullyrðingum.

Næsta spurning er þá -- hversu litlar þurfa líkurnar að vera svo við séum sátt?

\begin{center}\rule{0.5\linewidth}{0.5pt}\end{center}

\hypertarget{marktektarmuxf6rk}{%
\section{Marktektarmörk}\label{marktektarmuxf6rk}}

\textbf{Marktektarmörk} (\(\alpha\)) lýsa því einmitt hvar við drögum línuna, algengast er að miða við við \(\alpha\)= 0,05. \footnote{Undantekningar má t.d. finna í stjarnfræði þar sem við getum ekki tekið jafn mikla sénsa á að hafa rangt fyrir okkur.} Þessi tala (0,05) vísar til líkinda (5\%) og við erum þá að segja að við viljum að það séu undir 5\% líkur á að við höfum rangt fyrir okkur.

\hypertarget{marktektarpruxf3f}{%
\section{Marktektarpróf}\label{marktektarpruxf3f}}

\textbf{Marktektarpróf} reikna fyrir okkur líkur þess að fá tilteknar niðurstöður ef núlltilgáta væri í raun rétt og kallar þær \emph{p}-gildi. Ef við höfum ákveðið að draga mörkin við 5\% líkindi (0,05) þá munu allar niðurstöður þar sem \emph{p}-gildi er yfir 0,05 þýða að niðurstöður séu ómarktækar. \emph{P} gildi yfir 0,05 þýðir að það séu yfir 5\% líkur á að við höfum rangt fyrir okkur og við vorum jú búin að ákveða að draga mörkin við 5\%.

Athugum þó að 5\% líkur er ekki það sama og 0\% líkur. Það er, þó próf sé marktækt, þá \emph{gætum} við haft rangt fyrir okkur. Þannig skoðum aðeins \emph{hvernig} við getum haft rétt eða rangt fyrir okkur með því að gefa okkur eftirfarandi aðstæður í samhengi við dæmið okkar:

\begin{longtable}[]{@{}
  >{\raggedright\arraybackslash}p{(\columnwidth - 4\tabcolsep) * \real{0.1375}}
  >{\raggedright\arraybackslash}p{(\columnwidth - 4\tabcolsep) * \real{0.4375}}
  >{\raggedright\arraybackslash}p{(\columnwidth - 4\tabcolsep) * \real{0.4250}}@{}}
\caption{Tafla 1. Niðurstöður marktektarprófs borin saman við raunveruleikann.}\tabularnewline
\toprule()
\endhead
& Það er ekki munur á launum í þýði & Það er munur á launum í þýði \\
Marktækt & Höfnunarmistök (type I error) & Rétt niðurstaða - Afköst \\
Ómarktækt & Rétt niðurstaða & Fastheldnimistök (type II error) \\
\bottomrule()
\end{longtable}

\hypertarget{huxf6fnunarmistuxf6k}{%
\section{Höfnunarmistök}\label{huxf6fnunarmistuxf6k}}

\textbf{Höfnunarmistök (type I error)} er þegar við gerum þau \textbf{\emph{mistök}} að \textbf{\emph{hafna}} núlltilgátunni þegar hún er í raun rétt.

Skoðum hólf 1 í töflu 1: Hér er í raun engin munur á launum eftir menntun en niðurstöður okkar eru marktækar. Frá okkar sjónarhorni hugsum við eftirfarandi: \emph{P}-gildi er undir 0,05 --\textgreater{} próf er marktækt --\textgreater{} ef próf er marktækt höfnum við núlltilgátunni --\textgreater{} ef núlltilgátan segir að það sé engin munur, og hún er röng, þá hlýtur að vera munur! --\textgreater{} við tökum þá upp aðaltilgátu og ályktum að það sé í raun munur á launum eftir menntun.

Hér höfum við gert þau mistök að segja að það sé munur, þegar það er í raun engin munur.

\hypertarget{fastheldnimistuxf6k}{%
\section{Fastheldnimistök}\label{fastheldnimistuxf6k}}

\textbf{Fastheldnimistök (type II error)} er þegar við gerum þau \textbf{\emph{mistök}} að \textbf{\emph{halda í}} núlltilgátu sem er í raun röng.

Skoðum hólf 4 í töflu 1: Hér er í raun munur á launum eftir menntun en niðurstöður okkar eru ómarktækar. Við hugsum því: \emph{P}-gildi er yfir 0,05 --\textgreater{} prófið er ómarktækt --\textgreater{} ef próf er ómarktækt þá getum við ekki hafnað núlltilgátunni --\textgreater{} núlltilgátan segir að það sé engin munur --\textgreater{} ef ég get ekki hafnað núlltilgátunni, þá get ég ekki tekið upp aðaltilgátuna --\textgreater{} get ekki sagt að það sé munur á launum eftir menntun.

Hér höfum við gert þau mistök að segja að það sé ekki munur, þegar það er í raun munur.

\begin{center}\rule{0.5\linewidth}{0.5pt}\end{center}

\textbf{Hvort er verra; að gera höfnunarmistök eða fastheldnimistök?}

Hvorugt er gott en höfnunarmistök eru oft talin valda meiri skaða heldur en fastheldnimistök. Það byggir á þeim rökum að verra sé að hafa rangar upplýsingar heldur en að missa tímabundið af réttum upplýsingum. Það er, betra að segja ``ég er ekki viss'' heldur en að koma með fullyrðingu sem er röng.

Það eru þó vissulega til aðstæður þar sem fastheldnimistök eru verri. Dæmi um slíkt væri til dæmis ef aukaverkun tiltekins lyfs væri aukin hætta á heilablóðfalli. Í því tilviki myndum við frekar vilja \emph{err on the side of caution}. Við viljum frekar ganga út frá því að þessi hættulega aukaverkun sé raunverulegur möguleiki -- og hafa rangt fyrir okkur -- heldur en að hafa rangt fyrir okkur með því að segja að af lyfinu stafi engin hætta.

Íhugið eftirfarandi fullyrðingar:

\begin{itemize}
\item
  Þunglyndir eru líklegri til að sýna ofbeldisfulla hegðun.
\item
  Tiltekið lyf eykur líkur á heilablóðfalli verulega.
\item
  Hvernig gætum við yfirfært þessi tvö dæmi á fastheldni- og höfnunarmistök? Og hvor mistökin þætti okkur verri í hvoru dæmi fyrir sig?
\end{itemize}

\begin{center}\rule{0.5\linewidth}{0.5pt}\end{center}

\hypertarget{tuxfalkun-uxf3marktektar}{%
\section{Túlkun ómarktektar}\label{tuxfalkun-uxf3marktektar}}

Hvernig túlkum við ómarktækar niðurstöður? Skoðum hólf 3 í töflu 1, niðurstöður eru ómarktækar og við hugsum; \emph{p}-gildið er yfir 0,05 --\textgreater{} prófið er því ómarktækt --\textgreater{} við getum ekki hafnað núlltilgátu --\textgreater{} getum því ekki tekið upp aðaltilgátu --\textgreater{} við ályktum að það sé engin munur. Í þessu dæmi vitum við að það er í raun engin munur á launum eftir menntun, svo þessi ákvörðun er rétt.

Við getum þó ekki \textbf{staðfest núlltilgátu}, þó við náðum ekki að hafna henni, þá getum við heldur ekki fullyrt að hún sé endilega rétt. Ómarktækar niðurstöður segir bara að \emph{úrtakið okkar} náði ekki að sýna fram á nógu mikinn mun á launum eftir menntun m.t.t. úrtaksstærðar og annarra þátta, upp að því marki að niðurstöður þættu ólíklegar \textbf{ef það væri í raun engin munur}.

Fyrir það fyrsta er verulega ólíklegt að núlltilgátan sé virkilega \textbf{rétt} (þó við vorum að ímynda okkur slíkar aðstæður í hólfi 3 í töflu 1). Ef núlltilgátan segir að launamunur eftir menntun sé núll, þá þýðir það bókstaflega að munurinn þurfi að vera \textbf{núll}. Ekki núll komma eitthvað, ekki næstum því núll, það þyrfti raunverulega að vera \textbf{engin} munur. Þriðju breytu áhrif útaf fyrir sig gera það að verkum að það er oft \emph{einhver} fylgni á milli jafnvel ólíklegustu hluta.

Hugsum þetta út frá öðru sjónarhorni: Ef úrtakið okkar nær ekki að sýna fram á mun á milli hópa, þá þýðir það ekki að sá munur sé ekki raunverulega til staðar -- það þýðir \textbf{bara} að við náðum ekki að sýna fram á mun. Mögulega þyrfti úrtakið að vera stærra, mældur munur í úrtaki að vera meiri o.s.frv. svo að marktektarprófið væri í stöðu til að yfirhöfuð ná að staðfesta þann mun sem raunverulega er til staðar.

Þessi síðasti punktur leiðir að næstu pælingu, sem er; hversu mikill þarf munurinn að vera og hversu stórt þyrfti úrtakið að vera, svo að marktektarprófið ætti yfirhöfuð séns á að geta sýnt fram á þann mun sem raunverulega er til staðar?

\hypertarget{afkuxf6st}{%
\section{Afköst}\label{afkuxf6st}}

Hingað til höfum við verið að skoða mikið byggt frá því sjónarmiði hvort núlltilgátan sé rétt eða röng. Hér prufum við að snúa þessu við og hugsum dæmið út frá því að við vitum að \textbf{aðaltilgátan} sé rétt -- af einhverri ástæðu vitum við fyrir víst að það sé raunverulega munur á launum þeirra sem hafa og hafa ekki lokið háskólamenntun. Afköst eru einmitt hæfni prófs, að gefnum tileknum aðstæðum -- til að geta borið kennsl á þann mun sem er í raun og veru til staðar í þýði.

\textbf{Afköst} endurspegla líkur þess að við höfum rétt fyrir okkur þegar aðaltilgátan er rétt (það er í raun munur á hópunum). Við getum þó einnig hugsað um þetta sem líkur þess að gera \textbf{ekki fastheldnimistök}. Við getum reiknað afköst í höndunum \footnote{Við ætlum ekki að gera það.}, með forritlingum á netinu, skoðað sambærilegar rannsóknir og miðað við að finna a.m.k. eins mikið frávik og kemur fram í þeim \footnote{Við erum þá að skoða aðrar rannsóknir, þar sem viðfangsefni er svipað og úrtaksstærð er svipuð þeirri sem við ætlum okkur að safna til að fá hugmynd af því hve mikill munurinn er sem við þyrftum að finna til að sýna fram á marktekt.}, eða nota staðlaðar áhrifastærðir \footnote{Cohens \emph{d} er dæmi um staðlaða áhrifastærð -- þá er gefið viðmið um það hvaða áhrifastærð telst nógu mikil.}

\hypertarget{loka-puxe6lingar-um-marktekt.}{%
\section{Loka pælingar um marktekt.}\label{loka-puxe6lingar-um-marktekt.}}

Þegar hér er komið höfum við framkvæmt marktektarpróf, fengið \emph{p}-gildi og getum þar af leiðandi fjallað um niðurstöður okkar sem marktækar eða ómarktækar. Við erum því hér komin með skýrt svar við spurningunni \emph{er munur á launum eftir menntun?} Gefum okkur nú að prófið okkar sé marktækt og svarið því:

\textbf{Marktækur munur fannst á launum fólks eftir menntun þeirra}.

Byggt á þessari staðhæfingu gæti ég hugsað ``Glæsilegt -- þannig ef ég mennta mig get ég búist við umtalsvert hærri launum fyrir vikið - annars væru niðurstöður varla marktækar'' Eða hvað? Þessi túlkun gæti verið rétt en það eru þó nokkur atriði til að huga að.

``\emph{þannig ef ég mennta mig\ldots\ldots{}}''

\begin{itemize}
\item
  Marktækar niðurstöður þýða að það sé \textbf{að jafnaði} munur á launum hópanna. Það er ekki þar með sagt að þær niðurstöður muni eiga við um \textbf{alla} þó hún eigi við \textbf{að jafnaði}. Ef við skiptum fólki í tvennt eftir því hvort það hefur háskólamenntun eða ekki þá mætti segja að heilt yfir séu laun þessara hópa ólík. Fyrir tiltekin einstakling er sennilega munur - en ekki endilega.
\item
  Sambærilegt dæmi væru þær niðurstöður að \emph{``fullorðnir séu að jafnaði hærri en börn}''. Við vitum þetta vel en áttum okkur líka á því að það eru vissulega til börn sem eru hærri en margir fullorðnir.
\end{itemize}

``\emph{\ldots get ég búist við umtalsvert hærri launum fyrir vikið\ldots.}''

\begin{itemize}
\item
  Núlltilgátan okkar var \emph{H0}: meðallaun háskólamenntaðra = meðallaun óháskólamenntaðra. Það er engin munur á launum - þau eru þau sömu fyrir báða hópa.
\item
  Þær aðaltilgátur sem við prófum geta verið stefnutilgátur (einhliða) eða stefnulausar (tvíhliða). \textbf{Stefnulaus} aðaltilgáta segir bara ``það er munur'' án þess að tilgreina í hvaða átt munurinn er, svo háskólamenntaðir \emph{gætu} þess vegna haft lægri laun. \textbf{Stefnutilgáta} tilgreinir hins vegar í hvaða átt munurinn sé og væri þannig að prófa þá tilgátu ``háskólamenntaðir hafa hærri laun heldur en óháskólamenntaðir'' (eða öfugt).
\item
  Við þurfum að passa að túlka marktektina í samræmi við þær tilgátur sem voru prófaðar.
\end{itemize}

``\emph{\ldots.annars væru niðurstöður varla marktækar}''

\begin{itemize}
\tightlist
\item
  Það er alltaf hætta á að við höfum rangt fyrir okkur (séum að gera höfnunarmistök). Annað (en þó tengt) vandamál er að eftir því sem úrtakið er stærra þarf munurinn að vera þeim mun minni svo marktekt náist. Í smáum úrtökum þarf munurinn að vera meiri til að ná marktekt heldur en í stórum úrtökum (líkt og við sáum í dæmi við upphafi marktektarkafla)
\end{itemize}

Loka pælingin snýr að orðinu \emph{umtalsvert}. Marktekt segir bara \textbf{hvort} það sé munur en ekki \textbf{hve mikill} hann er - sem tekur okkur í næstu umfjöllun.

\hypertarget{uxf6ryggisbil}{%
\chapter{Öryggisbil}\label{uxf6ryggisbil}}

Niðurstöður okkar voru marktækar - við teljum því ólíklegt að munurinn sem fannst í úrtakinu sé tilkomin af tilviljun en vitum ekki hve \emph{merkilegur} sá munur er. Næsta spurning er því ``\emph{Hve mikill er launamunur þeirra sem hafa háskólamenntun og þeirra sem hafa hana ekki?}''

Raunin gæti verið sú að ef við berum saman þá sem hafa ekki háskólamenntun við þá sem hafa háskólamenntun, þá sé marktækur munur á laununum -- en að sá munur sé að jafnaði aðeins 10.000kr á mánuði. Þrátt fyrir að munurinn sé \emph{marktækur} þá er hann vissulega ekki \emph{merkilegur} -- enda svarar það varla kostnaði að lifa við skertar tekjur til fleiri ára, í þeim tilgangi að sækja sér menntun, ef afrakstur þess eru 10.000kr aukalega á mánuði í kjölfarið.

Gefum okkur skemmtilegri niðurstöður, að munurinn sé \emph{að jafnaði} 150.000 á mánuði. Þetta eru vissulega merkilegri niðurstöður en þær sem við fjölluðum um í dæminu á undan (þar sem munur var aðeins 10.000kr á mánuði).

Bæði dæmin lýsa \textbf{punktspá}, það næsta sem við komumst að vita hver munurinn sé í þýði er með því að athuga hver munurinn er í úrtakinu. Við vitum þó að ekki \textbf{allir} þeir sem hafa háskólamenntun fái ákkurat 150.000kr meira í laun á mánuði -- enda vissulega fleiri þættir sem spila inn í. Pælingin verður því ``hversu mikið gæti þetta spágildi rokkað til eða frá?''.

Þessari pælingu er svarað með \textbf{bilspá,} öryggisbil er dæmi um bilspá auk þess að vera sennilega sú mest notaða. Öryggisbil gefa okkur \textbf{vikmörk} í kringum punktspánna. Þegar hingað er komið vitum við að ``það er munur og hann er að jafnaði 150.000kr á mánuði''. Öryggisbil bætir við þessa túlkun; tökum tvö dæmi:

\begin{enumerate}
\def\labelenumi{\arabic{enumi}.}
\tightlist
\item
  Öryggisbil = {[}140.000 ; 160.000{]}

  \begin{itemize}
  \item
    Þetta myndi segja okkur að launamunur sé að jafnaði 150.000kr á mánuði - til eða frá 10.000kr.
  \item
    Hér er öryggisbilið frekar þröngt, punktspáin helst sú sama og það er lítil óvissa sem fylgir henni.
  \end{itemize}
\item
  Öryggisbil = {[}10.000 ; 290.000{]}

  \begin{itemize}
  \item
    Þetta myndi segja okkur að launamunurinn sé að jafnaði 150.000kr á mánuði - til eða frá 120.000kr.
  \item
    Hér er öryggisbilið vítt, punktspáin er sú sama en það er mikil óvissa sem fylgir henni. Munurinn gæti verið allt frá mjög ómerkilegum upp í umtalsverður. Launamunur gæti verið allt frá 10.000kr fyrir suma, upp í 290.000kr fyrir aðra.
  \end{itemize}
\end{enumerate}

Í seinna dæminu er óvissan töluvert meiri og niðurstöðurnar eru í raun að gefa okkur minni upplýsingar heldur en í fyrra dæminu. Seinna dæmið væri í raun að segja ``það gæti verið fínn munur, hann gæti líka verið mikill, eða enginn -- hver veit?'' Eftir stendur að ef öryggisbilið er mjög vítt, þá erum við litlu nær -- \emph{þrátt fyrir að niðurstöður séu marktækar}.

  \bibliography{book.bib,packages.bib}

\end{document}
